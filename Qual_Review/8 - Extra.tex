\section{Extra}

Things that have came up in Quals / were an important topic in the Stat Mech course.


\subsection{Equipotential Theorem}

As mentioned in the stat mech course there is a crude and precise definition:

\begin{itemize}
	\item Crude: $\langle E /$ particle $ \rangle = $ \# of degrees of freedom $* \frac{1}{2}kT$
	\item Precise: $\langle E /$ particle $ \rangle = $ \# of quadratic terms in the Hamiltonian $* \frac{1}{2}kT$
	      \begin{itemize}
		      \item Notes: The Hamiltonian is the energy of the system, KE + PE, and if there is only 
			  	momentum space there is no potential energy
		      \item Reminder: We can find the internal energy from the
	      \end{itemize}
\end{itemize}


\subsection{Magnetization}

\begin{itemize}
	\item Magnetic Moment ($\vec{\mu}$): measurement of the torque a system expaerinces in a $\vec{B}$
	      \begin{itemize}
		      \item From past quals it seems the magnetic moment is given
		      \item Ampèrian loop: $\vec{\mu} = I \vec{A}$, I is the current \& A is the area
		      \item Solenoid: $\vec{\mu} = N I \vec{A}$, where N is the turns
		      \item For spin particles: $\vec{\mu} = -g \mu_B \vec{S} / \hbar$
	      \end{itemize}
	\item Energy in a $\vec{B}$
	      \begin{align}
		      E = - \vec{\mu} \cdot \vec{B}
	      \end{align}
	\item Adiabatic Demagnetization: process used to cool systems to very low temps
	      \begin{itemize}
		      \item Apply strong $\vec{B}$ (this aligns the spins thus reducing the entropy)
		      \item Thermally isolate the system
		      \item Reduce the $\vec{B}$ but total entropy remains constant and temp drops
	      \end{itemize}
	\item Bohr Magneton ($\mu_B$): the magnetic moment of an electron due to its spin
\end{itemize}


\subsection{Density of State}