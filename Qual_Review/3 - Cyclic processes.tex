
\section{Cyclic processes}


\textcolor{violet}{Cyclic processes: thermodynamic process in hich a system returns to its initial 
thermodynamic state (i.e same U, P. V. T) after a process.
    \begin{itemize}
        \item aka $\Delta U = 0$
    \end{itemize}
}

\begin{itemize}
    \item We are reminded of the first law: $\Delta U = Q - W$
    \begin{itemize}
        \item meaning for a cyclic process  $\rightarrow \Delta = 0 \rightarrow Q = W$
        \item total heat added over the cycle = total work done by the system 
    \end{itemize}
    \item Most of this section refrenced Chapter 8 (Heat \& Work) from Kittel 
\end{itemize}


\subsection*{Energy \& Entropy Transfer Definitions}
\begin{itemize}
    \item \textcolor{violet}{heat is the transfer of energy to a system by thermal contact 
    with a reservoir }
    \item \textcolor{violet}{work is the transfer of energy to a system by a change in the 
    external parameters that describe the system }
    \item we want to convert heat to work (steam engine, combustion engine)
    \item bringing 2 systems together, the total energy is conserved but not the entropy 
    may increase
    \item it is defined that $dQ \equiv T dS$ which means $dW = dU - dQ = dU - TdS$
    \begin{itemize}
        \item $dS = 0 \rightarrow$ pure work 
        \item $dU = TdS \rightarrow$ pure heat
    \end{itemize}
\end{itemize}

\subsection*{Heat Engines: conversion of heat into work}
\begin{itemize}
    \item Carnot inequality
    \begin{itemize}
        \item work can be completely converted to heat but the inverse is not ture 
    \end{itemize}
    \item 
    \begin{itemize}
        \item 
    \end{itemize}
\end{itemize}

\subsection*{Heat \& Work at Constant Temp or Pressure }